\chapter{Application to Complex Problems}

There are some problems that cannot be easily modelled by boolean functions. This chapter will focus on applying the methods we have discussed to problems in image classification. Using the classical example of MNIST \cite{mnist}, an input image representing a handwritten digit from 0 to 9 is mapped to a probability distribution over all possible classifications. Deep learning proved to be very effective at this task \cite{mnist}. We hope to build an architecture that would be reasonably effective, while incorporating neurosymbolic elements that would allow us to interpret the learnt model.

We have mentioned mixed models many times in the previous chapter, but so far have not introduced an example of such. The language of first-order logic cannot describe boolean functions with multiple outputs. We define some primitive functions which do have multiple outputs, that can be composed to create more complex models. The function
$$
c : [0,1]^a \to [0,1]^b ;\ \vx \mapsto (c_1(\vx), \dots, c_b(\vx))
$$
represents a series of $b$ conjunctions $c_i$ over $\vx$, each learnt independently of each other. We define disjunctions $d$ similarly. $\psi$ will represent some composition of $c$'s and $d$'s (e.g., $d \circ c$ would capture boolean functions as elements of a strict superset of DNF.)

Let $m$ describe a classical MLP network, comprising of an alternating series of linear and non-linear layers, beginning and ending with a linear layer. We can define a mixed model MNIST classifier by
$$
M(\vx) = \text{softmax} \circ \psi \circ \sigma \circ m (\vx) 
$$

$\sigma \circ m_1$ isolates important features from the origin space, and ``encodes'' their \textit{presence} as boolean values in real logic $[0,1]$. $\psi$ performs some logical operations on the features generated by $m_1$, and outputs another set of features as elements of the space $[0,1]$. Finally, $\text{softmax}$ uses these features to generate a probability distribution over possible categories.

\todo{What if $\psi$ is the identity function?}

\todo{Actual test}

\todo{Analysis and improvements}